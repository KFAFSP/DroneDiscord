\chapter{Introduction}

% Mission statement.
This project aims to capture the essence of the \emph{Garry's Mod} gamemode \emph{Trouble in Terrorist Town} in the form of a community board game for \emph{3+} players.

% The game of TTT.
In \TTT{}, a group of players is thrown onto a limited map in an FPS shooter setting, where a randomly selected and undisclosed minority of players is tasked with killing all others.
While the majority faction is not aware of the identity of their opponents, they can combine close observations, deductive reasoning and interpersonal skills to determine and preventively kill the enemies.
Although outnumbered by a substantial factor, these enemies know of their partner's true identity, and are allowed secretive access to a reward system that provides them with advantages on the battlefield.
As engagements in the FPS world are fast-paced and sparse, and majority players are punished for wrongfully killing players, they are at a disadvantage.
Compensating for this fact, the majority players get a special, globally announced detective with access to a reward system, and are granted a time victory condition.

% Feature-richness of TTT.
\TTT{} features a very basic set of gameplay rules that can be expanded upon by countless additions to the in-game world, which do not necessarily have to interface with its core game mechanics.
The usage of new maps, addition of environmental hazards, introduction of new weapons, and customized rewards can enhance the gameplay experience without impacting any of the game mechanics directly.
However, it is also simple to make said mechanics more complex by adding new roles with specialized victory conditions to the game.
All these facts combined have resulted in \TTT{} becoming a staple of \emph{Garry's Mod} custom gamemode servers, and kept the game alive for years.

% Essence of TTT.
At its core, \TTT{} is based on deception and countermeasures to it.
The game should always keep players on their toes, doubting their fellow player's actions and statements, creating tension for an overall more violent, and consequently more fun, session.
Still, it should not be an entirely skill-based game --- players may improve their engagement outcomes and learn to see through their peer's lies, but elements of randomness and complexity must keep the game fair.

% DDD's goals.
\DDD{} aims to capture the same gameplay feel.
However, map-based board games are based largely around shared information, which defeats the purpose of \TTT{}.
In order to keep the elements of communication, deception and uncertainty in the game, \DDD{} introduces an anonymous voting mechanic that provides confidentiality for the minority's moves.
To cover-up this seemingly arbitrary mechanic, it is bolstered by a backstory that tries to engage the player in a scenario where victory conditions and playstyles feel more natural.