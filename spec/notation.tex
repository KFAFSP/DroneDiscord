% Introduction.
The following chapters define all the terms and rules in the current edition of the game.
Being a semi-formal specification, all game relevant terms require a highlighted definition, and all rules must be identified using a unique name.
Below is an example of how definitions of terms and rules are typeset in this book.\medskip

% Example definition.
An \term{example}\label{term:example}, also \term{sample}, is a specimen characteristic of its kind that illustrates a general rule.\medskip

% Example term reference.
Everywhere a \hyperref[term:example]{term} occurs, it is hyperlinked to its definition.\medskip

% Example inline rule.
0\S{}1\,-\,\ruleid{example.inline} dictates that inline rules start a paragraph with the rule number and name, and extend over the entire paragraph.\medskip

% Example rule list.
\begin{itemize}
    \item[0\S{}2]\ruleid{example.list}

    Rules in a list are preceded by their number and id, with the rule text in separate paragraphs.

    \item[0\S{}3]\ruleid{example.stmts}

    Rules might have more internal complexity, if that fits within a single logical scope, possibly referencing \term{well-defined game terminology} (see 0.1).

    \begin{stmts}
        \stmt{} Rules consisting of multiple statements receive a statement numbering.
        \stmt{} Statements are listed in order of precedence.
        \stmt{} Statements with higher indices overrule those with lower indices.
    \end{stmts}
\end{itemize}\medskip

% Example rule reference.
If necessary at any point, a rule like 0\S{}2 can be referred to by its number.