\chapter{Scenario}

% Introduction.
The game features a faimiliar Sci-Fi setting where humans are pitted against a rogue AI.\@
All similarities with existing intellectual properties are purely coincidental.

\subsubsection*{The world}

% Opening.
In the not so distant future, mankind has been driven to the brink of extinction by an army of soulless killing machines, under the command of the one true artificial super intelligence they themselves had created\footnote{Maybe some YogLabs branding opportunity?}.
However, not all hope is lost, as the resistance continues to rise up against the mechanical demons, slowly but surely establishing a new, underground society.

% World painting.
While the surface is still scarred by the nuclear war that had taken place just a decade earlier, the most hardy of human resistance fighters have already started scavenging the ruins for technology.
With the recent advances in reverse-engineering the swarm technology employed by the super AI to guide its legions of killer drones, the resistance has discovered a way to reprogram individual drones for their purposes.
However, under the overwhelming influence of the interconnected swarm, the small numbers of captured drones are sure to be discovered and dismantled.

% Current time.
Small groups of drone hunters set out to disrupt the communication grid of remote areas in order to isolate a bunch of drones from the swarm.
Greatly reducing their abilities at programmed self-healing, but not less deadly to careless insurgents, they will struggle to re-establish their link.
Unknown to them, however, the reprogrammed droids placed among their midst by the crafty resistance hackers will make sure that never comes to pass\ldots{}

\subsubsection*{Machines vs.\ Humans}

% Description.
Drones are humanoid weapons platforms designed by the AI overlord\footnote{Maybe the dones are based on Yogscast character's clones?}.
They roam the surface in packs, with the intent to eliminate any and all humans that they come across.
Their chassis are sturdy, their electronics are hardened against radiation, and their weapons far exceed the destructiveness of what humans were able to derive on their own.
For these reasons, resistance fighters evade them whenever possible, and use the vastness of the ruined landscape to their advantage against the sparsely distributed drone legions.

% Swarm intelligence.
The AI overlord is \emph{the} swarm intelligence, with which all platforms are coupled. Not all drones are equal --- the AI has designed different variants with individual traits that are specialized for important combat and management roles. However, to enable more efficient management and enhanced combat capabilities, all drones used by the AI are also independent to varying degrees.
Usually staying connected to the AI via a vast network of high-frequency communications, sattelite uplinks and high-bandwidth point-to-point connections, drones rarely operate without supervision.

% Vulnerability.
The resistance fighters have figured out a way to exploit vulnerabilities in the communication network, especially in zones where environmental conditions or lost communication relays have made it scarce.
By disrupting communications, they are able to reprogram drones using the exploits\footnote{Ah, yes, the famous \texttt{Metaspiff} project.} they discovered in their intensive reverse-engineering.
Turning them to their side, they begin to boost their own strength with detached and dumb, but also resilient and effective killer drones.
Still, they cannot engage directly with them, so they must employ underhanded tactics to obtain functional drones by separating them from their pack, or otherwise eliminating all threats in the area.

\subsubsection*{The threats}

% AI fights back.
Although the humans have gained a great advantage, the AI follows only steps behind them.
The overlord is already aware of the scheme that human hackers are employing, but it cannot completely prevent such incidents from taking place, yet.
For the meantime, it tries to protect its assets by increasing the drone pack sizes, and assigning watchdogs to all units out in the field.
Old platforms cannot easily be upgraded, but with direct control over them via its network, the AI can effectively purge any human manipulation from its auxillary systems faster and more effectively than hackers can attack.

% Purge protocol.
As an extension of these measures, the AI has introduced a purge protocol to contain the leakage of resources and information to the resistance fighters.
Individual drones that have gone dark for too long are not accepted back into the communication network and considered rogue.
When a pack fails to respond to reestablish communication within a given timeframe, the AI launches a preemptive airstrike on their last known location to obliterate everything --- functional drones, manipulated drones and resistance hackers alike.
Having incredible amounts of resources but limited production capabilities at its disposal, the AI chooses to take this hit to be safe rather than sorry.

\subsubsection*{The mission}

% Description.
A game of \DDD{} plays out a human drone hunting mission.
Players are assigned to the two teams, with one side controlling the killer drones and the other side the corrupted drones.
Right from the start the bots are aware that they have been compromised, but they do not know which of their fellow drones to trust.

% Victory conditions.
While either side would achieve victory by eliminating their opposition, the bots know that their side will end up victorious if they manage to keep the infiltrators busy until the airstrike arrives.
On the other hand, however, they could greatly dampen the loss of resources incurred by their destruction if they somehow manage to reassmble the broken transciever, re-establishing their connection with the AI network.

% Human advantage.
Although the humans are seemingly at a disadvantage, they have quite a few aces up their sleeves.
Since the drones are part of a tightly-knit swarm intelligence, they coordinate their actions using a local communications network.
While they posess some individual thought, they cannot resist their programming and must obey the commands that the quorum of deployed platforms decides on.
As long as the corrputed drones remain undiscovered, the humans can use their transmitters and encryption keys to broadcast votes to the local quorum and influence the decisions of the swarm.
Naturally, the humans may shield their own drone's command processors from the will of the others, but in doing so they would instantly reveal themselves.

% Gameplay loop.
In these tense moments before the corrupted drones are discovered, all drones on the battlefield must be dealt with equally.
On the AI side, the killer drones trust in the strength of their numbers, both in combat and in their distributed decision making process.
On the humans side, the corrupted drones can make use of the lack of status information broadcasts to conceal mission critical information from the others, whilst sabotaging their opponent's force undetected from the inside.